\documentclass[a4paper]{comcv}
\usepackage{setspace}
%spacing
\usepackage{titlesec}
\titlespacing\subsubsection{0pt}{5pt}{0pt}

\color{white}
\usepackage[english]{babel}
\fullname{}{Mira Welner}{}
\cvtitle{ }
\website{https://mirawelner.com}{mirawelner.com}
\email{miraewelner@gmail.com}
\github{https://github.com/MiraWelner}{GitHub}
\linkedin{https://www.linkedin.com/in/MiraWelner}{LinkedIn}

\begin{document}
\section{EDUCATION}
\begin{flushleft}
{\bf University of California, Davis \hfill September 2018 -- June 2022}
\\{Computer Science Engineering, BS \hfill {Overall GPA: 3.4 -- Major GPA: 3.5}}
\end{flushleft}

\section{CONFERENCES and PRESENTATIONS}
\noindent\small\textbf{Sparse Infrared Spectroscopy for Detection of Volatile Organic Compounds}\href{https://arxiv.org/abs/2506.20678}{   [Preprint]}\\
\indent Mira Welner, Andre Hazbun, Thomas Beechem

\noindent\small\textbf{Characterizing Pediatric Hand Grasps During Activities of Daily Living to Inform
Robotic Rehabilitation \\and Assistive Technologies }\href{https://pubmed.ncbi.nlm.nih.gov/36176073/}{  [Paper]}\\
\indent Marcus Battraw, Peyton Young, Mira Welner, Wilsaan Joiner, Jonathon Schofield\\
\indent \textit{International Conference on Rehabilitation Robotics (ICORR 2022)}

\noindent\textbf{A video game system to assess advanced control interfaces for pediatric prostheses  }\href{https://burningsilicon.dev/docs/schofield_presentation.pdf}{  [Poster]}\\
\indent Mira Welner, Jonathon Schofield\\
\indent \textit{33rd Annual UC Davis Undergraduate Research, Scholarship and Creative Activities Conference (URSCA 2022)}

\noindent\textbf{Unsupervised Identification of Materials with Hyperspectral Images } \href{https://ojs.aaai.org/index.php/AAAI/article/view/21708}{  [Paper]}\\
\indent Mira Welner\\
\indent \textit{Thirty-Sixth AAAI Conference on Artificial Intelligence (AAAI 2021)}

\noindent\textbf{Identification of Materials in Hyperspectral Images Using an Autoencoder and ReLU Activation } \href{https://youtu.be/CvCrZSAqwbI}{[Talk]}   \href{https://burningsilicon.dev/docs/rice_poster}{[Poster]}\\
\indent Mira Welner, Aswin Sankaranarayanan\\
\indent \textit{2021 Virtual Ken Kennedy AI and Data Science Conference}

\noindent\textbf{Identification of Materials in Hyperspectral Images Using a Convolutional Neural Network-based Autoencoder } \href{https://burningsilicon.dev/docs/purdue_poster}{[Poster]}\\
\indent Mira Welner, Aswin Sankaranarayanan\\
\indent \textit{2021 Purdue Virtual Undergraduate Showcase}

\noindent\textbf{Updating the National Ignition Facility Codebase from Java 8 to Java 11}\\
\indent Mira Welner, Lyle Beaulac, Mikhail Fedorov\\
\indent \textit{2019 Lawrence Livermore Laboratory Summer Scholar Poster Symposium}\\

\section{RESEARCH ROLES}
\combosection{Biomedical Research Contractor}{McGowan Institute of Regenerative Medicine}{March 2025 — Present}{
\begin{tightlist}
        \item Employed as a federal contractor by the Veterans Association to work with Professor DeMazumder and his lab
        \item Developing a digital twin screening system using a Wasserstein GAN with Gradient Penalty (WGAN-GP) and Graph Neural Network (GNN) to predict hidden cardiovascular disease from electronic health records.
        \end{tightlist}}
\combosection{Software Engineering Contractor}{Hermit Tech}{February 2025 — Present}{
\begin{tightlist}
        \item Designing an automatic data pipeline which takes in duckdb data file processes the data to create a SQL file which can be read by evidence.dev to create a dashboard on a publicly accessible site.
        \item Designing CSS and markdown for use in landing page and documents
        for client use.
\end{tightlist}}
\combosection{Bioinformatics Engineer}{Signature Diagnostics}{July 2023 -- November 2024}{
\begin{tightlist}
        \item Worked with Dr. Paul Cohen at Signature Diagnostics, an early-stage biomedical startup that develops non-invasive methods of prenatal screening.
        \item Conducted analyses on RNA-Seq data to determine which classification method and set of genes would yield the best classifier for various genetic diseases.
        \item Augmented a proprietary algorithm that served as a binary classifier with a data filtering algorithm, transforming the classifier into a multiclass classifier.
        \item Used RAG to assist LLMs in distinguishing between severe and mild forms of preeclampsia.
\end{tightlist}}
\combosection{Spectroscopy and Vision Science Researcher}{Purdue University}{August 2022 -- November 2024}{
\begin{tightlist}
        \item Collaborated with Professor Thomas Beechem at Purdue University's Mechanical Engineering department to develop a data-lean algorithm that processes spectroscopy data using non-negative matrix factorization to detect contaminants in mediums such as water.
        \item Wrote and developed figures for a publication describing the processing algorithm for which I am first author; currently in the process of editing and submitting it.
        \item Served as lead programmer in a mechanical engineering lab. Created a GitHub repository for the lab and instructed lab members on GitHub use.
\end{tightlist}}

\combosection{Machine Learning and Vision Science Undergraduate Researcher}{CMU}{June 2021 -- September 2021}{
\begin{tightlist}
        \item Collaborated with Professor Aswin Sankaranarayanan at CMU Image Science Labs to develop a modified autoencoder, which had the standard convolutional neural network encoder, but the decoder used matrix manipulation, resulting in the hyperspectral image being compressed into its three primary component spectra.
        \item Presented research at the AAAI Undergraduate Symposium and sole-authored a student paper accepted and presented at the AAAI Conference on Artificial Intelligence.
\end{tightlist}}
\vspace{\topsep}

\combosection{Robotics and Programming Undergraduate Researcher}{UC Davis}{September 2019 -- March 2022}{
\begin{tightlist}
        \item Designed a user study for young children utilizing a video game interface connected to a myoelectric detection system and Raspberry Pi 4. Collected and analyzed muscular behavior data using a MATLAB program.
        \item My research was included in a proposal that successfully earned the lab an NSF grant.
        \item Received a Provost Undergraduate Fellowship Award and made a poster that was accepted at the Annual UC Davis Undergraduate Research, Scholarship and Creative Activities Conference.
        \item Co-authored a publication submitted to the IEEE ICORR conference, responsible for designing figures and describing my portion of the programming.
\end{tightlist}}

\combosection{LLNL Summer Scholar}{National Ignition Facility}{June 2019 -- September 2019}{
\begin{tightlist}
        \item Updated and refactored the six-million-line Java codebase responsible for operating the National Ignition Facility at Lawrence Livermore National Laboratories.
        \item Developed and implemented unit tests for specific sections of the codebase that lacked adequate testing coverage.
\end{tightlist}}

\section{LEADERSHIP ROLES}
\combosection{UC Davis HyperLoop Team President}{}{September 2019 -- March 2021}{
\begin{tightlist}
        \item Led the UC Davis OneLoop team in the research, design, and manufacturing of the Davis pod for the annual HyperLoop competition.
        \item Developed the control system programming for the pod using Structured Text.
        \item Successfully competed in the 2018 OneLoop college competition, earning a spot among the top 21 teams selected to attend the event in Hawthorne.
\end{tightlist}}
\vspace{\topsep}
\end{document}
